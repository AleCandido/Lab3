\documentclass[10pt,a4paper]{article}

\usepackage[utf8]{inputenc}
\usepackage[T1]{fontenc}	
\usepackage[italian]{babel}
\usepackage{amsmath}
\usepackage{amsfonts}
\usepackage{amssymb}
\usepackage{graphicx}

\usepackage[left=2cm,right=2cm,top=2cm,bottom=2cm]{geometry}
\geometry{a4paper}

\usepackage{booktabs} % for much better looking tables
\usepackage{verbatim}
\usepackage{subfig} % make it possible to include more than one captioned figure/table in a single 

\usepackage{fancyhdr} % This should be set AFTER setting up the page geometry
\pagestyle{fancy} % options: empty , plain , fancy
\renewcommand{\headrulewidth}{0pt} % customise the layout...
\lhead{}\chead{}\rhead{}
\lfoot{}\cfoot{\thepage}\rfoot{}

%%% SECTION TITLE APPEARANCE
\usepackage{sectsty}
%\allsectionsfont{\sffamily\mdseries\upshape} % (See the fntguide.pdf for font help)
% (This matches ConTeXt defaults)
% pacchetti che mi fanno schifo ma uso lo stesso (Bob è scemo...)
\usepackage[cdot, thickqspace]{SIunits}
% macro che mi piacciono
\def\code#1{\texttt{#1}}


\title{Esercitazione 4: Amplificatore a transistor}

\author{Gruppo BE \\ Alessandro Candido, Roberto Ribatti}
\date{\today}
\begin{document}
\maketitle

\section{Scopo e strumentazione}
Lo scopo dell'esperienza è realizzare e caratterizzare un amplificatore a transistor, usando un transistor NPN 2N1711.

La strumentazione usata è quella presente sul banco di lavoro, più il suddetto transistor.

\section{Montaggio del circuito e punto di lavoro}
Il circuito è stato realizzato usando i componenti riportati in Tabella~\ref{tab:componenti}

\begin{table}[h!]
\centering
\begin{tabular}{c|c|c}
$R_1 = \unit{177.1 \pm 1.5}{\kilo\ohm}$ & $R_2 = \unit{18.06 \pm 0.15}{\kilo\ohm}$ & $R_C = \unit{9.93 \pm 0.09}{\kilo\ohm}$ & & $R_E = \unit{986 \pm 9}{\ohm}$ \\
$C_{IN} = \unit{230 \pm 10}{\nano\farad}$ & $C_{OUT} = \unit{98 \pm 6}{\nano\farad}$ & $C_E = \unit{100 \pm 10\%}{\micro\farad}$\\
\end{tabular}
\caption{Valori dei componenti utilizzati per la realizzazione dell'amplificatore}
\label{tab:componenti}
\end{table}

Ad eccezione dell'ultima parte dell'esperienza non è stata usata la resistenza $R_{es}$.

\subsection{Punto di lavoro}
Si sono misurate corrente e tensione di quiescenza per determinare il punto di lavoro. Si riporta anche la tensione $V_{CC}$ erogata dall'alimentatore.
\begin{table}[h!]
\centering
\begin{tabular}{c|c|c}
$V_{CE}^Q = \unit{8.21 \pm 0.05}{\volt}$ & $I_C^Q = \unit{1.058 \pm 0.009}{\milli\ampere}$ & $V_{CC} = \unit{19.85 \pm 0.11}{\volt}$
\end{tabular}
\end{table}

Mentre i valori attesi sono\footnote{exp ad apice sta ad indicare che sono i valori attesi}:

\begin{table}[h!]
\centering
\begin{tabular}{c|c}
$V_{CE}^{exp} = \unit{7.64 \pm 0.26}{\volt}$ & $I_C^{exp} = \unit{1.229 \pm 0.026}{\milli\ampere}$
\end{tabular}
\end{table}

Per calcolarli si è usato che $V_{BE} = \unit{625 \pm 4}{\milli\volt}$, cioè quanto abbiamo misurato.

\subsection{Tensioni ai terminali del transistor}
Le tensioni misurate sui terminali del transistor sono le seguenti.

\begin{table}[h!]
\centering
\begin{tabular}{c|c|c|c}
$V_B = \unit{1.692 \pm 0.009}{\volt}$ & $V_E = \unit{1.070 \pm 0.006}{\volt}$ & $V_{BE} = \unit{625 \pm 4}{\milli\volt}$ & $V_{C} = \unit{9.28 \pm 0.06}{\volt}$
\end{tabular}
\end{table}

Mentre per i valori attesi si è ottenuto:

\begin{table}[h!]
\centering
\begin{tabular}{c|c|c}
$V_B^{exp} = \unit{1.837 \pm 0.022}{\volt}$ & $V_E^{exp} = \unit{1.212 \pm 0.23}{\volt}$ & $V_{C}^{exp} = \unit{ \pm }{\volt}$
\end{tabular}
\end{table}

\section{Risposta a segnali sinusoidali a frequenza fissa}



\section{Risposta in frequenza}

\section{Aumento del guadagno}

\end{document}
