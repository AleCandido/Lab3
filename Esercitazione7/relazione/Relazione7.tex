% !TeX spellcheck = it_IT
\documentclass[10pt,a4paper]{article}

\usepackage[utf8]{inputenc}
\usepackage[T1]{fontenc}	
\usepackage[italian]{babel}
\usepackage{amsmath}
\usepackage{amsfonts}
\usepackage{amssymb}
\usepackage{graphicx}

\usepackage[left=2cm,right=2cm,top=2cm,bottom=2cm]{geometry}
\geometry{a4paper}

\usepackage{booktabs} % for much better looking tables
\usepackage{verbatim}
\usepackage{subfig} % make it possible to include more than one captioned figure/table in a single 

\usepackage{fancyhdr} % This should be set AFTER setting up the page geometry
\pagestyle{fancy} % options: empty , plain , fancy
\renewcommand{\headrulewidth}{0pt} % customise the layout...
\lhead{}\chead{}\rhead{}
\lfoot{}\cfoot{\thepage}\rfoot{}

%%% SECTION TITLE APPEARANCE
\usepackage{sectsty}
%\allsectionsfont{\sffamily\mdseries\upshape} % (See the fntguide.pdf for font help)
% (This matches ConTeXt defaults)

% pacchetti che mi fanno schifo ma uso lo stesso (Bob è scemo, ma anche Ale...)
\usepackage[cdot, thickqspace, squaren]{SIunits}
% il miglior pacchetto che potessi desiderare
\usepackage{float}
% macro che mi piacciono
\def\code#1{\texttt{#1}}


\title{Esercitazione 7: Amplificatore operazionale, usi non lineari}

\author{Gruppo BE \\ Alessandro Candido, Roberto Ribatti}
\date{\today}
\begin{document}
\maketitle

\section{Scopo e strumentazione}

\section{Discriminatore}
Si è montato il circuito in \figurename{~\ref{circuito_discriminatore}} e se ne è studiata la risposta a segnali sinusoidali di frequenza e intensità variabili. In \figurename{~\ref{fig:discriminator}} è mostrato il comportamento ad una frequenza di $\sim \unit{1}{k\hertz}$ ed una tensione di ingresso di $\sim\unit{4}{\volt}$, è visibile l'onda sinusoidale in ingresso e l'onda quadra generata in uscita dall' op-amp. Poiché l'op-amp è usato in modalità invertente il comportamento è paragonabile ad un circuito logico NOT.

\subsection{Misura della tensione di offset}
Si è proceduto a studiare in maggior dettaglio il punto di zero-crossing e a misurare la tensione di offset $V_{OS}$, ovvero la tensione del segnale sinusoidale $V_{IN}$ nell'istante in cui il segnale in output attraversa lo zero ($V_{OUT}=0$). In figura \figurename{~\ref{fig:vos}} è visibile il punto di zero crossing. La misura ha restituito il valore di $V_{OS} = \unit{150 \pm 5}{\milli\volt}$

\subsection{Conservazione GBW}
Per elevate frequenze è atteso, per la conservazione del prodotto guadagno-banda che i guadagno dell'op-amp diminuisca. Per piccoli segnali in ingresso e alte frequenze è atteso che il segnale in output non arrivi a saturazione e sia visibile quindi come un onda sinusoidale.
In \figurename{~\ref{fig:GMW}} è visibile la risposta del circuito ad un segnale di $\sim\unit{100}{k\hertz}$ e un'intensità $V_{IN}= \unit{156 \pm 5}{\milli\volt}$ e come atteso il segnale è sinusoidale.
Si è misurata la risposta a varie frequenze a parità di tensione di input, i dati raccolti sono riportati in appendice in \tablename{~\ref{tab:GBW}}. Si è fittata la conservazione della prodotto banda guadagno, i grafici e i risultati del fit sono riportati di seguito:


\subsection{Slew rate}
Per frequenze medio/alte e grandi segnali lo slew-rate produce una notevole distorsione dell'onda quadra limitanto la velocità di salita/discesa del segnale. Glie effetti dello slew-rate sulla risposta del circuito sono chiari in \figurename{~\ref{fig:slewrate_razzista}}


\section{Amplificatore di carica}


\section{Trigger di Schmitt}

\section{Multivibratore astabile}

\pagebreak
\section{Appendice: Dati acquisiti}
Si riportano qui le tabelle dei dati usati per i fit e i grafici.

%\centering
%\begin{figure}[h!]
%	\begin{minipage}[t]{0.33\textwidth}
%		\resizebox{1\textwidth}{!}{
%		\input{../tabelle/tab_inv_amp_gain.txt}}
%		\captionof{table}{Guadagnio amplificatore invertente}
%		\label{tab:inv_amp_gain}
%	\end{minipage}
%	\begin{minipage}[t]{0.33\textwidth}
%		\resizebox{1\textwidth}{!}{
%		\input{../tabelle/tab_inv_amp_f_domain.txt}}
%		\captionof{table}{Risposta in frequenza amplificatore invertente}
%		\label{tab:inv_amp_f_domain}
%	\end{minipage}
%	\begin{minipage}[t]{0.33\textwidth}
%		\resizebox{1\textwidth}{!}{
%			\input{../tabelle/tab_gain_bandwidth.txt}}
%		\captionof{table}{Banda/Guadagno}
%		\label{tab:gain_bandwidth}
%	\end{minipage}
%\end{figure}

%\begin{figure}[h!]
%	\centering
%	\resizebox{0.7\textwidth}{!}{
%	\input{../tabelle/tab_low_pass.txt}}
%	\captionof{table}{Dati relativi al circuito integratore}
%	\label{tab:lowpass}
%\end{figure}

%\begin{figure}[H]
%	\centering
%	\resizebox{0.7\textwidth}{!}{
%	\input{../tabelle/tab_high_pass.txt}}
%	\captionof{table}{Dati relativi al circuito derivatore}
%	\label{tab:highpass}
%\end{figure}



\end{document}
