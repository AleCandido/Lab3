% !TeX spellcheck = en_US
\documentclass[10pt,a4paper]{article}

\usepackage[utf8]{inputenc}
\usepackage[T1]{fontenc}	
\usepackage[italian]{babel}
\usepackage{amsmath}
\usepackage{amsfonts}
\usepackage{amssymb}
\usepackage{graphicx}

\usepackage[left=2cm,right=2cm,top=2cm,bottom=2cm]{geometry}
\geometry{a4paper}

\usepackage{booktabs} % for much better looking tables
\usepackage{verbatim}
\usepackage{subfig} % make it possible to include more than one captioned figure/table in a single 

\usepackage{fancyhdr} % This should be set AFTER setting up the page geometry
\pagestyle{fancy} % options: empty , plain , fancy
\renewcommand{\headrulewidth}{0pt} % customise the layout...
\lhead{}\chead{}\rhead{}
\lfoot{}\cfoot{\thepage}\rfoot{}

%%% SECTION TITLE APPEARANCE
\usepackage{sectsty}
%\allsectionsfont{\sffamily\mdseries\upshape} % (See the fntguide.pdf for font help)
% (This matches ConTeXt defaults)

% pacchetti che mi fanno schifo ma uso lo stesso (Bob è scemo...)
\usepackage[cdot, thickqspace]{SIunits}
% macro che mi piacciono
\def\code#1{\texttt{#1}}


\title{Esercitazione 5: Transistor JFET}

\author{Gruppo BE \\ Alessandro Candido, Roberto Ribatti}
\date{\today}
\begin{document}
\maketitle

\section{Scopo e strumentazione}
Studiare le caratteristiche e realizzare un amplificatore con il JFET a canale N 2N3819.
La strumentazione usata è quella presente sul banco di lavoro, più il suddetto transistor.

\section{Studio del funzionamento del JFET}

\paragraph{LED} Il LED si accende e si spegne alla tensione $V_{GS} \sim \unit{1.89}{\volt}$. Infatti al di sopra di una certa tensione gate-source ($V_{GS}$) il canale è da considerarsi chiuso (pinch-off) e il transistor va in interdizione. Al di sotto della tensione di pinch-off a corrente inizia a scorrere nel canale, portando in conduzione anche il LED.S

\section{Montaggio amplificatore}

\section{Misure a frequenza fissa}

\section{Impedenza d'ingresso}

\section{Aumento del guadagno}


\end{document}
