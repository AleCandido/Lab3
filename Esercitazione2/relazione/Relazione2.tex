\documentclass[10pt,a4paper]{article}

\usepackage[utf8]{inputenc}
\usepackage[italian]{babel}
\usepackage{amsmath}
\usepackage{amsfonts}
\usepackage{amssymb}
\usepackage{graphicx}

\usepackage[left=2cm,right=2cm,top=2cm,bottom=2cm]{geometry}
\geometry{a4paper}

\usepackage{booktabs} % for much better looking tables
\usepackage{verbatim}
\usepackage{subfig} % make it possible to include more than one captioned figure/table in a single 

\usepackage{fancyhdr} % This should be set AFTER setting up the page geometry
\pagestyle{fancy} % options: empty , plain , fancy
\renewcommand{\headrulewidth}{0pt} % customise the layout...
\lhead{}\chead{}\rhead{}
\lfoot{}\cfoot{\thepage}\rfoot{}

%%% SECTION TITLE APPEARANCE
\usepackage{sectsty}
%\allsectionsfont{\sffamily\mdseries\upshape} % (See the fntguide.pdf for font help)
% (This matches ConTeXt defaults)

% pacchetti che mi piacciono
\usepackage[cdot, thickqspace]{SIunits}

\title{Esercitazione 2: Circuito RC - Filtri passivi}
\author{Gruppo BE \\ Alessandro Candido, Roberto Ribatti}
\date{\today} 

\begin{document}
\maketitle

\section{Scopo e strumentazione}
Lo scopo dell'esperienza è:
\begin{itemize}
\item verificare il comportamento in frequenza dei filtri RC posti singolarmente e a cascata (l'uscita dell'uno connessa con l'ingresso dell'altro);
\item dimensionare opportunamente un filtro passa-basso al fine di migliorare il rapporto segnale rumore in un caso dato.
\end{itemize}

In particolare per quanto riguarda il primo punto si è analizzato un filtro passa-basso e una configurazione a cascata, in cui il primo stadio era un passa-basso e il secondo un passa-alto. Per il secondo punto si aveva un segnale a $\unit{2}{\kilo\hertz}$ e il rumore a $\unit{20}{\kilo\hertz}$, e si considerava un carico a valle di $\sim \unit{100}{\kilo\ohm}$.

La strumentazione usata è quella presente sul banco di lavoro.
%così suona altamente poco riproducibile, ma dobbiamo trovare una formula per evitare di ripetere ogni volta l'elenco stupido di oscilloscopio, alimentatore, etc.

\section{Filtro passa-basso}

\subsection{Dimensionamento}
Si è dimensionato il circuito tenendo conto dei seguenti criteri:
\begin{itemize}
\item massimizzare il rapporto segnale-rumore;
\item ottenere un segnale abbastanza intenso.
\end{itemize}
Cercare di preservare l'intensità del segnale in ingresso è motivato dal fatto che, se il segnale venisse troppo attenuato dal filtro, gli strumenti per misurarlo introdurrebbero essi stessi un rumore non trascurabile, o comunque sarebbero non trascurabili altri tipi di contributi.

A questo fine si è fissata la frequenza di taglio a $\unit{200}{\hertz}$, infatti minore è la frequenza di taglio maggiore è il rapporto segnale rumore, per cui si otterrebbe un rapporto massimo per una frequenza di taglio nulla. Ma in questo modo si attenuerebbe eccessivamente il segnale, non rispettando il secondo criterio.

%Inserire figura rapporto segnale-rumore

A $\unit{200}{\hertz}$ infatti il guadagno è $\sim \unit{-20}{\deci\bel}$ per il segnale, per cui un segnale in ingresso di $\sim \unit{20}{\volt}$ (come quello fornito dal generatore di forme d'onda) viene attenuato a $\sim \unit{2}{\volt}$. In questo modo l'errore introdotto dalla lettura sull'oscilloscopio non è troppo influenzato dal fondo che si osserva in assenza di segnale, la cui intensità stimata è pari a $\sim \unit{2}{\milli\volt}$, quindi lo $0.1\%$ del segnale in uscita.

Per cui si è scelto per il filtro la resistenza $R = \unit{776 \pm 7}{\ohm}$ e il condensatore $C = \unit{1.0 \pm 0.1}{\milli\farad}$. La resistenza è stata scelta anche in modo tale da rendere poco influente la caduta di potenziale sull'impedenza di uscita del generatore. Il valore del carico nella configurazione usata per le misure era $R_{c} = \unit{96 \pm 9}{\kilo\ohm}$

Inoltre si è verificato che il condensatore scelto avesse un'impedenza trascurabile rispetto al carico, in modo da rendere la risposta del circuito indipendente dal valore esatto del carico, ammesso che rimanga dell'ordine di grandezza fissato (o maggiore).

\subsection{Risposta in frequenza}
Dalla teoria sappiamo che il guadagno del filtro passa basso è $|A_v|=1/\sqrt{1+(f/f_0)^2}$ dove $f_0 = 1/(2 \pi RC)$ è detta frequenza di taglio. A questa frequenza il filtro RC ha un guadagno di $\unit{-3}{\deci\bel}$.

Nel nostro caso a partire dal valore misurato di $R$ e dal valore nominale del condensatore $C$, il valore atteso è $f_0 = \unit{200 \pm 20}{\hertz}$ (dove l'incertezza è quasi totalmente determinata dal condensatore).

\paragraph{} Un modo veloce per misurare sperimentalmente questa grandezza è quindi quello di variare la frequenza con continuità fino a trovare  un rapporto $V_{out}/V_{in}$ pari a $1/\sqrt{(2)}$ (ovvero $\unit{-3}{\deci\bel}$). La misura effettuata in questo modo ha prodotto una stima di $f_1 =\unit{208 \pm 10}{\hertz}$.

\paragraph{} Un'altro modo per misurare la frequenza di taglio è eseguire due fit lineari a bassa ed alta frequenza (rispetto a $f_0$ attesa). Calcolando il punto di intersezione e la relativa incertezza (tenendo debitamente conto della covazianza associata alle misure prodotte dai fit) si otterrà la frequenza cercata. I risultati dei fit sono stati: 
\begin{itemize}
	\item prima retta (risposta a bassa frequenza):	 $a=\unit{-0.7 \pm 0.7}{\deci\bel/decade}$,  $b=\unit{2.1 \pm 2.3}{\deci\bel}$ e $c_{ab} = -0.976$, rispettivamente coefficiente angolare, intercetta e coefficiente di correlazione.
	\item seconda retta (risposta ad alta frequenza):	 $c=\unit{-18.2 \pm 0.3}{\deci\bel/decade}$ , $d=\unit{93.6 \pm 2.6}{\deci\bel}$ e $c_{cd} = -0.989$, rispettivamente coefficiente angolare, intercetta e coefficiente di correlazione.
\end{itemize}
L'andamento è quello previsto: la prima retta ha coefficiente angolare e intercetta entrambi compatibili con $0$, come atteso, e la seconda retta ha un'inclinazione di $\sim \unit{-20}{\deci\bel / decade}$.

Il calcolo dell'intersezione produce $f_2 =\unit{ 193 \pm 16}{\hertz}$.
Il grafico dell'intersezione delle rette fittate è riportato di seguito.
\begin{figure}[h]
	\centering
	\includegraphics[width=0.7\textwidth]{../grafici/fit_rette.pdf}
	\caption{Grafico dell'intersezione delle rette}
\end{figure}

\paragraph{} La frequenza di taglio può anche essere ottenuta direttamente attraverso il fit della funzione di trasferimento $|A_v|$. Le misure effettuate sono riportate di seguito (insieme a quelle eseguite per il punto precedente).


\begin{figure}[h!]
	\centering
		\input{../tabelle/tab_Bode_Lowpass_800ohm.txt}
		\captionof{table}{Dati raccolti}
\end{figure}
\begin{figure}[h]
		\centering
		\includegraphics[width=0.8\textwidth]{../grafici/fit_Bode_Lowpass_800ohm.pdf}
		\caption{Grafico delle misure e del fit}
\end{figure}

La frequenza di taglio fittata è $f_3 = \unit{229 \pm 4}{\hertz}$

\subsection{Risposta al gradino}
Si è misurato il tempo di carica del condensatore nel circuito in esame, dando in ingresso una funzione a gradino, ottenuta con un'onda quadra di bassa frequenza.\footnote{Se il periodo è abbastanza lungo si può assumere che il condensatore si carichi e si scarichi completamente.}

\begin{figure}[h!]
	\centering
	\includegraphics[width=0.5\textwidth]{../oscilloscopio/raise_time.jpg}
	\caption{Tempo di carica del condensatore}
	\label{fig:raise}
\end{figure}

Il valore ottenuto per il tempo di salita del segnale (tra il 10\% e il 90\% del massimo) è $\unit{1.62 \pm 0.06}{\milli\second}$, corrispondente ad una frequenza di $f_4 = \unit{216 \pm 7}{\hertz}$

\subsection{Domande}

\paragraph{Impedenza d'ingresso} L'impedenza d'ingresso è $Z_{in} = R + Z_c(\omega) = R + 1/i\omega C$, quindi:
\begin{itemize}
\item a bassa frequenza l'impedenza diverge;
\item alla frequenza di taglio $Z_{in} = R(1-i)$;
\item ad alta frequenza l'impedenza tende a $R$.
\end{itemize}

\paragraph{Resistenza di carico} L'effetto dell'inserimento di una resistenza di carico è cambiare l'impedenza equivalente del parallelo col condensatore (che in assenza di carico è data dal solo condensatore). In particolare in modulo l'impedenza del parallelo diminuisce, aggiungendo inoltre una parte reale, indipendente dalla frequenza, e quindi modifica i comportamenti asintotici.
Inserendo un carico di $\unit{10}{\kilo\ohm}$ il contributo della resistenza nell'impedenza del parallelo aumenterebbe, e quindi andrebbe tenuto di conto anche a frequenze più basse rispetto al caso con $\unit{100}{\kilo\ohm}$ (ovviamente più basse di circa un fattore $10$).

\section{Filtro passa-banda}

\subsection{I stadio: passa-basso}

Si è montato il filtro passa basso come richiesto con una resistenza $R_1 = \unit{3.25 \pm 0.04}{\kilo\ohm}$ (misurata con il multimetro) e un condensatore $C_1=\unit{10 \pm 1}{\nano\farad}$ (valore nominale).

Si è verificato che il filtro seguisse l'andamento generale previsto e che avesse guadagno $\sim 1$ a frequenza $<<f_0$, dove $f_0$ è la frequenza di taglio (in questo caso $f_0 = \unit{4.9 \pm 0.5}{\kilo\hertz}$).

Si è misurata la frequenza di taglio sia cercando la frequenza per cui il guadagno fosse $\unit{-3}{\deci\bel}$ che eseguendo un veloce fit della funzione di trasferimento.
I risultati sono stati:
\begin{itemize}
	\item $f_0 = \unit{4.2 \pm 0.1}{\kilo\hertz}$ con il primo metodo,
	\item $f_0 = \unit{5.10 \pm 0.14}{\kilo\hertz}$ con il secondo metodo.
\end{itemize}
I dati raccolti e il grafico del fit sono qui riportati:

\begin{figure}[h!]
	\centering
	\input{../tabelle/tab_Fast_fit_lowpass.txt}
	\captionof{table}{Dati raccolti}
\end{figure}

\begin{figure}{h!}
	\centering
	\includegraphics[width=0.8\textwidth]{../grafici/fit_Fast_fit_lowpass.pdf}
	\caption{Grafico delle misure e del fit}
\end{figure}

\subsection{II stadio: passa-alto}
Per il passa alto si è proceduto nella stessa maniera del punto precedente. Si sono usate $R_2=\unit{3.23 \pm 0.4}{\kilo\ohm}$ e $C=\unit{100 \pm 10}{\nano\farad}$.
In questo caso i risultati delle misure sono stati:
\begin{itemize}
	\item $f_0 = \unit{517 \pm 10}{\kilo\hertz}$ con il primo metodo,
	\item $f_0 = \unit{499 \pm 13}{\kilo\hertz}$ con il secondo metodo.
\end{itemize}

I dati raccolti e il grafico del fit sono qui riportati:

\begin{figure}[h!]
	\centering
	\input{../tabelle/tab_Fast_fit_highpass.txt}
	\captionof{table}{Dati raccolti}
\end{figure}
	
\begin{figure}[h!]
	\centering
	\includegraphics[width=0.8\textwidth]{../grafici/fit_Fast_fit_highpass.pdf}
	\caption{Grafico delle misure e del fit}
\end{figure}
\subsection{Configurazione a cascata}

\paragraph{Frequenze di taglio} Le frequenze di taglio misurate attraverso l'intersezione delle rette di fit sono $f_l = \unit{270 \pm 22}{\hertz}$ e $f_l = \unit{8.5 \pm 0.5}{\kilo\hertz}$.

\begin{figure}[h!]
	\centering
	\includegraphics[width=0.8\textwidth]{../grafici/fit_passabandar.pdf}
	\caption{Grafico del guadagno del filtro passa-banda, fittato con rette}
\end{figura}

\paragraph{Guadagno di centro banda} Il guadagno di centro banda, misurato con lo stesso fit usato per individuare le frequenze di taglio, è pari a $A_0 = \unit{-0.007 \pm 0.029}{\deci\bel}$.

\paragraph{Spiegazione}

\paragraph{Scelta di $R1$ e $R2$} Si poteva scegliere $R1$ e $R2$ in modo che il rapporto fra i due fosse approssimativamente nullo, ad esempio scegliendo $R1$ più piccolo oppure $R2$ più grande.
Infatti il guadagno complessivo può essere scritta come il prodotto dei guadagni dei due stadi per un fattore moltiplicativo. Se $A_1, A_2$ sono i guadagni del primo e del secondo stadio, allora:
\begin{equation*}
A_{tot} = A_1 A_2 \frac{1}{1 + \frac{R_1}{R_2} A_1 A_2}
\end{equation*}


%sui nomi possiamo sempre discutere dopo, per ora sono nomi significativi, se poi per correttezza o estetica vuoi cmabiarli decidiamo insieme
\end{document}
