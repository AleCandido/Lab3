\documentclass[10pt,a4paper]{article}

\usepackage[utf8]{inputenc}
\usepackage[italian]{babel}
\usepackage{amsmath}
\usepackage{amsfonts}
\usepackage{amssymb}
\usepackage{graphicx}

\usepackage[left=2cm,right=2cm,top=2cm,bottom=2cm]{geometry}
\geometry{a4paper}

\usepackage{booktabs} % for much better looking tables
\usepackage{verbatim}
\usepackage{subfig} % make it possible to include more than one captioned figure/table in a single 

\usepackage{fancyhdr} % This should be set AFTER setting up the page geometry
\pagestyle{fancy} % options: empty , plain , fancy
\renewcommand{\headrulewidth}{0pt} % customise the layout...
\lhead{}\chead{}\rhead{}
\lfoot{}\cfoot{\thepage}\rfoot{}

\newcommand\ohm{\Omega}
\newcommand\Hz{\text{Hz}}
\newcommand\dB{\text{dB}}
\newcommand\K{\text{~k}}
\newcommand\M{\text{~M}}
\newcommand\m{\text{~m}}

%%% SECTION TITLE APPEARANCE
\usepackage{sectsty}
%\allsectionsfont{\sffamily\mdseries\upshape} % (See the fntguide.pdf for font help)
% (This matches ConTeXt defaults)

\title{Esercitazione 2: Circuito RC - Filtri passivi}
\author{Gruppo BE \\ Alessandro Candido, Roberto Ribatti}
\date{\today} 

\begin{document}
\maketitle

\section{Scopo e strumentazione}
Lo scopo dell'esperienza è:
\begin{itemize}
\item verificare il comportamento in frequenza dei filtri RC posti singolarmente e a cascata (l'uscita dell'uno connessa con l'ingresso dell'altro);
\item dimensionare opportunamente un filtro passa-basso al fine di migliorare il rapporto segnale rumore in un caso dato.
\end{itemize}

In particolare per quanto riguarda il primo punto si è analizzato un filtro passa-basso e una configurazione a cascata, in cui il primo stadio era un passa-basso e il secondo un passa-alto. Per il secondo punto si aveva un segnale a $2\K\Hz$ e il rumore a $20\K\Hz$, e si considerava un carico a valle di $\sim 100\K\ohm$.

La strumentazione usata è quella presente sul banco di lavoro.
%così suona altamente poco riproducibile, ma dobbiamo trovare una formula per evitare di ripetere ogni volta l'elenco stupido di oscilloscopio, alimentatore, etc.

\section{Filtro passa-basso}

\subsection{Dimensionamento}
Si è dimensionato il circuito tenendo conto dei seguenti criteri:
\begin{itemize}
\item massimizzare il rapporto segnale-rumore;
\item ottenere un segnale abbastanza intenso.
\end{itemize}
Cercare di preservare l'intensità del segnale in ingresso è motivato dal fatto che, se il segnale venisse troppo attenuato dal filtro, gli strumenti per misurarlo introdurrebbero essi stessi un rumore non trascurabile, o comunque sarebbero non trascurabili altri tipi di contributi.

A questo fine si è fissata la frequenza di taglio a $200~\Hz$, infatti minore è la frequenza di taglio maggiore è il rapporto segnale rumore, per cui si otterrebbe un rapporto massimo per una frequenza di taglio nulla. Ma in questo modo si attenuerebbe eccessivamente il segnale, non rispettando il secondo criterio.

A $200~\Hz$ infatti il guadagno è $\sim -20~\dB$ per il segnale, per cui un segnale in ingresso di $\sim 20V$ (come quello fornito dal generatore di forme d'onda) viene attenuato a $\sim 2V$. In questo modo non l'errore introdotto dalla lettura sull'oscilloscopio non è troppo influenzato dal fondo che si osserva in assenza di segnale, la cui intensità stimata è pari a $\sim 2\m V$, quindi lo $0.1\%$ del segnale in uscita.

Per cui si è scelto per il filtro la resistenza $R = $ e il condensatore $C = $. La resistenza è stata scelta anche in modo tale da rendere poco influente la caduta di potenziale sull'impedenza di uscita del generatore. 

\subsection{Risposta in frequenza}
Dalla teoria sappiamo che il guadagno del filtro passa basso è $|A_v|=1/\sqrt{1+(f/f_0)^2}$ doce $f_0 = 1/(2 \pi RC)$ è detta frequenza di taglio. Nel nostro caso $f_0 =$ A questa frequenza il filtro RC ha un guadagno di $-3 \dB$.

Un modo veloce per misurare sperimentalmente questa frequenza è quindi quello di variare la frequenza con continuità fino a trovare  un rapporto $V_{out}/V_{in}$ pari a $1/\sqrt{(2)}$ (ovvero $-3\dB$). La misura effettuata in questo modo ha 

\subsection{Risposta al gradino}

\subsection{Temp:domande teoriche}
%nome temporaneo, decidiamo poi insieme

\section{Filtro passa-banda}

\subsection{I stadio: passa-basso}

\subsection{II stadio: passa-alto}

\subsection{Configurazione a cascata}

%sui nomi possiamo sempre discutere dopo, per ora sono nomi significativi, se poi per correttezza o estetica vuoi cmabiarli decidiamo insieme
\end{document}
